\section{}
The window size specifies how many bytes can be transferred without
acknowledgement from the other device. 
When selecting the window size, it is important to consider that a larger window
size allows for greater variability in timing, but also requires the reveiver to
be prepared to receive larger bursts of data bits. A larger window size is more
desireable for high speed and reliable connections, like optical fibre. However,
selecting too large of a window size means that the receiver may not have enough
space in its receive buffer which would lead to loss in data. 

However,
if the connection is not as reliable, a smaller window size may be more
desireable, since it requires acknowledgement that the data was recieved more
often. This can avoid repeated transmissions of data. However, selecting too
small of a window size can lead to inefficient utilzation of the connection. 

When a window size of 0 is received, the sender stops sending data. 

The window size is advertised by the receiver, so that the sender knows how many
bytes it can send before waiting for an acknowledgement.

When selecting the maximum size segment size, it is important to consider that
it specifies how much data the receiver can receive in a single TCP segment.
Having a smaller maximum segment size will reduce IP fragmentation, however,
this comes with the cost of higher overhead.
It is typically determined by the operating system of the communicating devices,
and each device can use a different maximum segment size. It is
announced during the TCP handshake in the SYN, and is not a negotiated
parameter. 


