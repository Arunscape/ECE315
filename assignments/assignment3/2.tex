\section{}
\begin{comment}
  Multiple Nonpreemptive Loops with Interrupts (cont’d)
• Advantages:
– Guaranteed minimum polling frequency for each
input or device
– Guaranteed maximum response time for polled
inputs and devices
– External events are handled promptly and
deterministically using hardware-triggered interrupts.
• Disadvantages:
– Some devices may be polled faster than necessary,
which is wasteful of the processor’s time.
– The one software thread has less cohesion because
it now handles more than one loop. A state variable
keeps track of the timer ISR calls and the active loop.

\end{comment}


\subsection*{Similarities}
\begin{enumerate}
  \item both use harmonic intervals to move between different sections of code
  \item each polled input or device is assigned to one loop or state
  \item loops/states not giving up CPU until the code is done processing
\end{enumerate}

\subsection*{Differences}
\begin{enumerate}
  \item separate blocks of code represented as a different loop vs. a different
    states
  \item interupts for responding to external events, (Multiple nonpreemptive
    loops) provide fast and deterministic response to external events vs. giving
    up CPU to the kernel when the task is complete, meaning external events are
    only processed when the state associated with that peripheral is run
    (periodically state-driven)
  \item advancing from state to state happens at predictable intervals, but with
    multiple nonpreemptive loops with interrupts, an interrupt can vary the time
    it takes to switch between tasks
\end{enumerate}
