\section{}
A simple implementation of co-operative multitasking can be done using state
machines. Each task would be called in the main loop, similar to how tasks are
called in one nonpreemptive loop. However, each task would also modify a global
variable which stores the current state of the state machine. 
For example, each task could have global integer variable, which holds its
current state. Then, in each task, it would check what state it is currently in,
and run some code, update its corresponding state variable, and return, to yield
CPU to the next task.


Idle time can be implemented by creating an idle task that executes \texttt{NOP}
in an infinite loop until a hardware timer signals the start of the next looping
interval. 
