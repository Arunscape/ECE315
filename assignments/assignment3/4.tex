\section{}
The number of teeth a gear has determines how many steps there are in a full
rotation. For example, a gear with four teeth will have 4 steps in a
360\textdegree rotation. Therefore, the step angle for the one gear would be
360\textdegree/4 = 25\textdegree. The formula is 360\textdegree divided by the
number of teeth. Now, in the stepper motor, there is a rotor and stator, and the
rotor has more teeth than the stator. Because the rotor has more teeth than the
stator, we can see both visually and mathematically that the individual step
angle for the gear would be smaller than the stator. Because these gears rotate
in opposite directions, the overall rotation would be the larger step angle of
the stator, subtracted by the smaller step angle of the rotor. We subtract the
step angle of the rotor, because it goes in the opposite direction of the
stator. So, overall, we get \[\frac{360^\circ}{n_s} - \frac{360^\circ}{n_r}\]
where \(n_s\) is the number of teeth of the stator, and \(n_r\) is the number of
teeth of the rotor. This matches with the formula described in slide 10-10.
