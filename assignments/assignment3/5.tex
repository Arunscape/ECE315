\section{}

The stepper motor control algorithm could be modified as follows

\begin{adjustbox}{width=2\textwidth,center}
	\digraph{motor}{
	// rankdir=LR;
	size="8,5";
	overlap=false;
	// node [shape = doublecircle];
	node [shape = circle];
	begin [label= "", shape=none,height=.0,width=.0];
	begin -> Stationary [label = "reset"];
  Stationary -> Stationary [label="destination_position == current_position"];
  Stationary -> Accelerating [label="destination_position != current_position"];
  Accelerating -> Accelerating [label = "Not near minumum stopping distance, and
  not at slew rate"];
  Accelerating -> Decelerating [label="Approaching min. stopping dist."];
  Accelerating -> Slewing [label="Not near minimum stopping distance, and have
  reached the slew rate"];
  Slewing -> Slewing [label="Not near minimum stopping distance"];
  Slewing -> Decelerating [label = "Approaching min. stopping dist."];
  Decelerating -> Decelerating [label="Not yet at destination position"];
  Decelerating -> Stationary [label="destination_position == current_position"];
  Accelerating -> Decelerating [label="Above new slew rate" color=green];
  Slewing -> Accelerating [label="Below new slew rate" color=green];
  Slewing -> Decelerating [label="Above new slew rate" color=green];
  Decelerating -> Slewing [label="Reached the new slew rate" color=green];

	}
\end{adjustbox}

Basically, in the four code snippets, a new global variable would be added next
to the other three (
\texttt{present\_state},
\texttt{present\_position},
\texttt{step\_period}
) called \texttt{slew\_rate}.

In the acceleration and slewing code, there would be a check to see if the
slewing rate changed. If the current slewing rate is faster than what the new
slewing rate is supposed to be, then the state should switch to the decelerating
state. Similarly, in the decelerating and slewing code, if the new slewig rate
is faster than the current slewing rate, it should transition to the
accelerating state. 
