\section{}
\subsection*{}
The system will enter any of the low power modes below when a \texttt{STOP}
instruction is executed. Depending on the \texttt{WCR[LPMD]} bits, the MCU will
enter either wait, doze, or stop mode.
For each of these modes, power is saved by idling the CPU with no active
cycles, powering down the system, and stopping all internal clocks. The coldfire
core is also disabled in each of the low-power modes. 

A wake-up event will cause the MCU to exit any of these low power modes, and
return to run mode. A wake-up event can be any type or reset, or any valid,
enabled interrupt request. To exit from these low power modes with an interrupt,
the interrupt request's priority should be higher than the value programmed in
\texttt{WCR[LPMD]}, and higher than the value programmed in the interrupt
priority mask (I) field of the core’s status register. Additionally, the
interrupt request should be from a source that is not masked in the interrupt
controller’s interrupt mask register, and it should have been enabled at the
module of the interrupt’s origin.

\subsubsection*{wait} 
Wait mode saves power by stopping the CPU and memory clocks until a wake-up
event is detected. Peripherals can be programmed to continue operating, and can
also generate interrupts which will cause the CPU to exit from wait mode. The
wait mode is entered when the \texttt{STOP} instruction is executed, with the
\texttt{WCR[LPMD]} bits having a value of \texttt{10}.

%A wake-up event will cause the MCF54415 to exit wait mode.
\subsubsection*{doze} 
Doze mode is similar to wait mode, in that it affects the
processor in the same way as in wait mode, however, some peripherals define
individual operational characteristics in doze mode. Peripherals continuing to
run and having the capability of producing interrupts may cause the CPU to exit
the doze mode and return to run mode. Stopped peripherals restart operation on
exit from doze mode, as defined for each peripheral
The doze mode is entered when the \texttt{STOP} instruction is executed with the
\texttt{WCR[LPMD]} bits having a value of \texttt{01}.

\subsubsection*{stop}
Stop mode is also similar to wait and doze mode, in that it affects the
processor the same way. However, all system clocks are stopped, and peripherals
cease operation.
The stop mode is entered when the \texttt{STOP} instruction is executed with the
\texttt{WCR[LPMD]} bits having a value of \texttt{11}.

%The SRAM, Power Management, Chip Configuration Module, System Control Module,
%GPIO, Edge Port, eDMA Controller, FlexBus Module, DDR Controller, USB OTG, USB
%Host, SSI, Programmable Interrupt Timers, DMA Timers, 
