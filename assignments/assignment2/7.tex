Manchester codes were widely used for magnetic recording on 600 bpi computer
tapes, and was used for the early physical layer ethernet standards.
Today, they are used in consumer IR protocols, RFID, and NFC communication.

It is a method of transmitting bits which allows the receiver to synchronize
with the sender.

\footnotetext{\url{https://www.allaboutcircuits.com/technical-articles/manchester-encoding-what-is-it-and-why-use-it/}}


\includegraphics*[width=\textwidth]{manchester.png}

4b/5b Encoding used for fast ethernet (100Mbps), and USB power delivery. It also
allows for synchronization between the reciever and the sender.
\footnotetext{\url{https://topic.alibabacloud.com/a/4b-5b-coding-principle_8_8_31582667.html}}

\includegraphics*[width=0.5\textwidth]{4b5b.jpg}




In terms of advantages over Manchester encoding, 4B/5B more efficiently utilizes
the link. With Manchester encoding, the data rate is cut in half relative to the
bandwidth of the data signal.  In comparison, for 4B/5B, the signalling rate is
1.25 times the data rate. Additionally, 4B/5B encoding allows for some error
detection, if a code recieved is not in the table.
