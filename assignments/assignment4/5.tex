\section{}

When establishing the connection, the client sends its sequence number (
a value between 0 and the 32-bit maximum unsigned integer). When the server
receives the client's request, it replies with its own sequence number, and an
acknowledgement indicating the next segment number it is expecting from the
client. The client receives the response from the server, and sends a
(pure) acknowledgement indicating the next segment number it is expecting from
the server. At this point, the TCP connection is established, and the server is 
ready to listen for a request from the client.
The client is then free to send a request with its current sequence number to
the server, and when the server receives the segment, it responds with an
acknowledgement indicating the next segment it is expecting to receive. This
process is repeated until the connection is closed. 

For example, let's say that a client wanted to connect to a server.

\begin{enumerate}
  \item Client sends SYN with its sequence number. Let's say this number is 100
  \item The server then sends a SYN with its own sequence number. Let's say this
    number is 200
  \item The server also sends an acknowledgement to the client, saying it's
    expecting the next segment to be 101
  \item The client receives the server's SYN, and responds with a pure
    acknowledgement saying it expects the next segment number to be 201.
  \item the client sends a segment with segment number 101, and increments its
    own sequence number
\end{enumerate}
