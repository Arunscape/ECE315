\section{}
When the data arrives out of order from the perspective of the receiver, it is
stored, and presented to the process in the correct order. 

Out-of-order packets are detected by the receiving networking hardware by
checking the sequence portion of the TCP header. This is a 32-bit field which
comes after first 32 bits. (That is, bits 32-63) in the header.
The purpose of the sequence field is store a number which allows the
bytes which are sent to be ordered. The starting number is agreed upon
when setting up the connection, and subsequent segments have an increasing
segment number in their respective header fields. 

If a later
segment arrives before the one which is expected, the receiver can hold on to it
while it waits for the next segment which it is expecting to arrive. 
Say, for example the receiver was waiting for a segment, and the segment number
for argument's sake is 901. Due to different routes taken by the datagrams,
segment 903 arrives first instead. The receiver would send an acknowledgement
saying it is expecting segment 901. This lets the sender know that a datagram
was received, but it was not the right one. (If 3 duplicate acknowledgements
arrive, it is assumed the datagram was lost and needs to be re-transmitted).
Then, say segment 902 arrives. This is again out of order, and the receiver
would send another acknowledgement asking for segment 901. Finally, when segment
901 arrives, the receiver has segments 901, 902, and 903, so it sends an
acknowledgement asking for segment 904, which lets the sender know that the
previous segments have been received.
