\section{}

DHCP and BOOTP cannot use TCP, because TCP requires both nodes to have unique
IP addresses. When a DHCP client for example first starts up, it does not know
which host or hosts it wants to connect to, so it sends a discover broadcast
message to all hosts on the network. This broadcasting activity is not supported
by TCP, because this activity has a one-to-many relationship as opposed to a
one-to-one relationship. The purpose of DHCP and BOOTP is to obtain an IP
address. That is, when the process is done, the computer will then have an IP
address, meaning the computer does not have an IP address in the middle of the
DHCP or BOOTP process. 

Both work in a similar fashion, since DHCP is an extension of BOOTP.

  First, there is a broadcast message, in which a client asks for an IP address.

  Then, the  broadcast message is picked up, and the BOOTP/DHCP server responds
    with information that the client needs such as the client’s IP address,
    subnet mask, default gateway address,  the IP address and host name of the
    BOOTP/DHCP server. This is also broadcasted, since the client does not have 
    an IP address next. 

    The client then picks up this information, and processes it. 
