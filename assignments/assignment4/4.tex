\section{}
FIFO buffers are used at the sending and receiving ends to make sure that 
the buffers can hold data if there are short-term mismatches in the rate that
the data is transferred. The order in which the data is sent/received is
preserved, and the data does not have to be re-transmitted. 

In contrast, flow control is used to avoid long-term mismatches in the data
rates. The receiver gives feedback to the sender which lets the sender know
whether it should slow down or speed up the rate at which it is sending data. 
It can also tell the sender to stop sending data if it is overwhelmed, and
resume sending data. 

FIFO buffers and flow control are used together decouple the data rates between
two communicating computers. 

Without data buffers, and only using flow control, the system would be less
tolerant of short-term mismatches. More re-transmissions of data would be
required, since small fluctuations in the sending and receiving rate of data
means that the sender may not send the data at the correct rate, and the receiver
may not be ready to receive data at the correct rate 
which would result in some incoming data being lost.

Without flow control, if the data rate is too fast, the receiver's buffer could
get filled and we'd have a similar issue, where the receiver is not ready to
receive data, and the lost data has to be re-transmitted. Alternatively, if
the data rate is too slow, the sender's buffer could fill up, and the receiver's
buffer would be mostly empty, being underutilized since a higher data rate would
be more ideal. 
