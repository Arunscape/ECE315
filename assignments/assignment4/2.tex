\section{}

\subsection*{Similarities}
\begin{itemize}
  \item From the user perspective there is not a huge difference between the two
    strategies of displaying dynamic webpages
  \item both methods run scripts on computers to display dynamic content
\end{itemize}

\subsection*{Differences}
\begin{itemize}
  \item where the code is run. As the name suggests, client-side scripts run on
    the user's computer, instead of the server. Server-side scripts run on a
    remote server, instead of the user's computer. 
\end{itemize}

\subsection*{Server-side scripting}
\subsubsection*{Advantages}
\begin{enumerate}
  \item the user does not have to install plugins (for example, Flash)
  \item scripts are hidden from the point of view of the user. They only see the
    HTML that is outputted. This can help with protecting intellectual property,
    especially if the code is not obfuscated
  \item In general, load times are quicker since the website and its content is
    already generated/pre-computed so the client does not have to do any
    additional processing.
  \item More control over what code is executed. The client does not have the
    ability to modify the scripts that are run.
\end{enumerate}
\subsubsection*{Disadvantages}
\begin{enumerate}
  \item more investment is needed in server compute resources, since scripts are
    processed on the server side
  \item does not scale well with many users. As your website gets more traffic,
    you need to spend more resources on server hardware compared to client side
    scripting
\end{enumerate}

\subsection*{Client-side scripting}
\subsubsection*{Advantages}
\begin{enumerate}
  \item more interactivity, since user actions can be processed immediately
    without requiring a trip to the server and back
  \item less network usage, since user actions are processed locally, meaning
    the server has to send and receive less data
  \item less server compute resources required compared to server side, since
    the processing is done client-side. As the website gains more users, not as
    much processing resources need to be added (although bandwidth would still
    be an issue)
\end{enumerate}
\subsubsection*{Disadvantages}
\begin{enumerate}
  \item Depending on the scripting language chosen, the user's browser may not
    support it, meaning they would not be able to interact with the website
  \item Different browsers may have different implementations of the scripting
    language, which adds some overhead for the developers, since more cases need
    to be considered and as a result, more testing to support users with
    different browsers
  \item Load times may be slower since the client must run the script to
    generate the site and its contents
  \item Less control over what gets executed. A malicious client might be able
    to modify the code and run it in an unexpected way
  \item the client's computer may be too slow to run the scripts in a reasonable
    time, leading to a poor user experience
  \item a client who has not refreshed the website in a long time may be running
    an older script which was previously updated
  \item proprietary code can potentially be exposed to users if it is not
    obfuscated
\end{enumerate}

