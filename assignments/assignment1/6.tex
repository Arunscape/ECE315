Buffer overflow happens then elements are added to a buffer (implemented as an
array) and they go beyond the buffer's capacity. This can be dangerous, since it
allows for arbitrary data to potentially overwrite other data which might be in
use, or code. This can be a security issue which allows an attacker to change
data that they should not have access to, or to modify code that will be run.
This can also result in system instability and crashes, since unexpectedly
changing data results in undefined behaviour. 

Embedded systems are generally more vulnerable to buffer overflow, since they
usually do not offer the same protections as a fully fledged operating system.

In C, these problems can be avoided by using \texttt{strncpy()} as opposed to
\texttt{strcpy()}, and \texttt{fgets()} instead of \texttt{gets()}. Bounds
checking can also help with preventing buffer overflows. If the programmer makes
sure that they never attempt to go out of the bounds of the buffer, overflow
won't happen. 
