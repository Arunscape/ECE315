The system would fall behind on its real-time guarantees if the amount of idle
time was reduced below a safe percentage of the CPU' execution time. For
externally initiated events, too little idle time would mean not enough excess
CPU capacity so that the worst-case bursts of event-handling workloads would not
be able to be handled within the maximum response time variability
specifications. 

For internally initiated events, not enough idle time would mean that the
effects of internal sources of timing variability (e.g., variability in the
execution time of compiled software, cache memory and virtual memory,and DMA
activity) would not be absorbed and therefore not hidden. 
%The timing for internally initiated events should be determined by H/W timers.

You would start to see problems like the system being less responsive, and
longer wait times for tasks to complete.
