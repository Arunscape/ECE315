\subsection{Major elements of this architecture}
The super short version:
\begin{enumerate}
  \item it's single threaded
  \item it's non-preemptive
  \item there is a main loop, where most of the processing is done
  \item Interrupt Service Routines (ISR) respond to critical events
\end{enumerate}

\noindent
A little more in-depth:
\begin{enumerate}
  \item foreground-background systems:\\
    Background processing: The main loop can meet less challenging timing
    constraints through normal instruction execution.\\ 
    Foreground processing: ISRs provide faster
    response for critical events (assuming interrupt hardware is present)
  \item sleep mode with interrupts: This allows the microcomputer to go in a
    low-power sleep state that can be woken up with external interrupt signals that exceed a specified interrupt priority level.
  \item one non-preemptive loop with no interrupts:
    A number of input sources or devices are polled, it iterates at a frequency
    that is at least the fastest required polling frequency. The longest
    possible execution time in the loop must be less than the loop period,
    and the frequency is determined by a hardware timer. 
    A waiting sub-loop at the end of the main loop safely uses
    up excess time at the end of each iteration of the main loop.
  \item one non-preemptive loop with interrupts:
    We enhance a single non-preemptive loop with a few interrupts to ensure that
    hardware-triggered events are serviced promptly and deterministically.  It
    can enable data processing by signalling a flag or semaphore that allows
    software outside of the ISR (which should be short) to proceed later after a
    small delay, and it also keeps the processor lightly loaded so that the
    worst-case execution of ISRs does not compromise the real-time performance
    of the main processing loop.
  \item multiple non-preemptive loops with interrupts:
   Two or more non-preemptive loops which should be implemented using one
   software thread are used that iterate at different harmonic periods.
   Interrupts are used to provide fast and deterministic response to external
   events.  Executions of the multiple loops are enforced using a single
   hardware timer to ensure timing accuracy.  Most data processing is done
   outside the ISRs in the one main software thread, so the ISR should be kept
   short.  This keeps the processor lightly loaded so that even the worst-case
   execution of ISRs does not compromise the real-time performance of the
   multiple processing loops.  Also, each polled input or device is assigned to
   the one loop that iterates just fast enough to meet its real-time constraints
  \end{enumerate}

  \subsection{How it simplifies meeting these real-time specifications}
\begin{enumerate}[label={\alph*)}]
  \item maximum response times:
    Using one non-preemptive loop gives us a guaranteed maximum response time. 
    Using one non-preemptive loop with interrupts, we get predictable maximum
    response time for both polled devices and external events.
    Having a predictable maximum response time makes it easy to determine if
    we will meet the maximum response time criteria

  \item maximum variability in the response times: 
    A long main loop might actually result in overly variable response times to
    input events, but in general, this should be lower in a non-preemptive loop
    architecture compared to a preemptive one since exceptions and internal
    signals would contribute to the variability, but for a preemptive
    architectures,  exceptions, internal signals, and preemptive tasks
    contribute to the variability in response times. 

  \item accuracy of repetition frequencies:\label{c}
    A non-preemptive system is likely to have a much more predictive repetition
    frequency compared to a preemptive one, since in the non-preemptive system,
    this repetition frequency is based on the loop, while in a preemptive
    system there is overhead with running the OS among other factors which
    result in less predictable repetition frequencies compared to a
    non-preemptive system.

  \item maximum variability in the repetition frequencies:
    For the same reason as \ref{c} above, the maximum variability in the
    repetition frequencies will overall be less variable and more consistent in
    a non-preemptive system compared to a preemptive system due to the looping
    architecture in the non-preemptive system being much more predictable than
    running an operating system that has extra overhead.

  \item control of degredation behaviour:
    In preemptive systems, the control of degredation behaviour is actually
    implicitly handled through prioritization, but for non-preemptive systems
    this behaviour has to be explicitly implemented. Luckily, ISRs can be used
    to provide a faster response for critical events such as if the battery is
    nearly exhausted. ISRs are supposed to not compromise the real-time
    performance of the system, and the small overhead they add does not compare
    the the overhead of a fully-fledged operating system that uses preemptive
    scheduling.
\end{enumerate}
