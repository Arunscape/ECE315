\subsection{Major elements of this architecture}
\begin{enumerate}
  \item it's single threaded
  \item foreground-background systems:\\
    Background processing: The main loop can meet less challenging timing
constraints through normal instruction execution.\\
Foreground processing: Interrupt service routines provide faster response for
critical events (assuming interrupt hardware is present)
  \item sleep mode with interrupts: This allows the microcomputer to go in a
    low-power sleep state that can be woken up with external interrupt signals that exceed a specified interrupt priority level.
  \item one non-preemptive loop with no interrupts:
    A number of input sources or devices are polled, it iterates at a frequency
    that is at least the fastest required polling frequency. The longest
    possible execution time in the loop must be less than the loop period,
    and the frequency is determined by a hardware timer. 
    A waiting sub-loop at the end of the main loop safely uses
    up excess time at the end of each iteration of the main loop.
  \item one non-preemptive loop with interrupts:
    We enhance a single nonpreemptive loop with a few interrupts to ensure that
    hardware-triggered events are serviced promptly and deterministically.  It
    can enable data processing by signalling a flag or semaphore that allows
    software outside of the ISR (which should be short) to proceed later after a
    small delay, and it also keeps the processor lightly loaded so that the
    worst-case execution of ISRs does not compromise the real-time performance
    of the main processing loop.
  \item multiple non-preemptive loops with interrupts:
   Two or more nonpreemptive loops which should be implemented using one
   software thread are used that iterate at different harmonic periods.
   Interrupts are used to provide fast and deterministic response to external
   events.  Executions of the multiple loops are enforced using a single
   hardware timer to ensure timing accuracy.  Most data processing is done
   outside the ISRs in the one main software thread, so the ISR should be kept
   short.  This keeps the processor lightly loaded so that even the worst-case
   execution of ISRs does not compromise the real-time performance of the
   multiple processing loops.  Also, each polled input or device is assigned to
   the one loop that iterates just fast enough to meet its real-time constraints
  \end{enumerate}
`
\subsection{How it simplifies meeting these real-time specifications}
\begin{enumerate}[label={\alph*)}]
  \item .
  \item 
\end{enumerate}
