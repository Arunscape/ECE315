A pointer can be created in the following ways:
\begin{enumerate}
  \item You can ask the OS to allocate a region of
    memory using \texttt{malloc()}, \texttt{calloc()}, or \texttt{realloc()},
    and give you a pointer to that region of memory. It is up to the programmer
    to check if NULL was returned and deal with error handling.
  \item The \texttt{\&} operator is used to obtain a pointer to an existing
    piece of data
  \item math is done on an existing pointer, and the result is stored as a new
    pointer. (e.g. \texttt{char* new\_ptr = other\_ptr + 8;})
  \item A pointer can be declared but uninitialized. A so-called wild pointer
    can point to anywhere in memory and therefore should not be used. A pointer
    should be initialized with a value before use.
  \item A pointer can be declared and initialized to a specific address (e.g.
    \texttt{void* ptr = 0x93784928;})
\end{enumerate}

Throughout a pointer's life, it can be modified and overwritten. An example of a
pointer being modified would be a typed pointer being incremented or decremented
before or after they are used. (e.g. \texttt{*(ptr++)}.  Void pointers, which
can be used to point to anything can be casted to a different type before being
modified (e.g. \texttt{ptr = (char* ptr)}) as well. Also, a pointer can be
overwritten by simply assigning another pointer value to the existing pointer.
Pointers are also usually dereferenced in their lifetime using the * operator.
This allows for the data that the pointer points to, to be accessed. 

When the object a pointer points to no longer needs to be used, a responsible C
programmer will call \texttt{free()} to recycle the memory.  Once the block of
memory that a pointer points to is freed using the \texttt{free()} function, the
pointer can be safely discarded. Actually, the pointer should no longer be used,
because the pointer now points to a region of memory that has been recycled.
This is called a dangling pointer. Failure to call \texttt{free()} after the
object is no longer required results in what is known as a memory leak.

