\subsection{Cloud Computing}

As per ISO/IEC 17788:2014 :
\begin{displayquote}
  ``
Cloud computing is a paradigm for enabling network access to a scalable and elastic pool of shareable physical or
virtual resources with self-service provisioning and administration on-demand. The cloud computing paradigm is
composed of key characteristics, cloud computing roles and activities, cloud capabilities types and cloud service
categories, cloud deployment models and cloud computing cross cutting aspects
that are briefly described in [clause 6 of ISO/IEC 17788:2014].
''
\end{displayquote}
Advantages:
\begin{itemize}
  \item The cloud computing provider takes care of purchasing and maintaining
hardware, not the user
\item Allows for flexibility. The user can scale services as they need
  on-demand. That way, a user only pays for the resources they use. For example,
  the user can request more computing resources when their service has more
  traffic, but when there is less traffic, the user can scale down their
  operations to save cost
\item Reduces barrier of entry for deploying an application. One can deploy a
  toy project they've been working on for a few dollars, without investing in
  the up-front cost of purchasing a mainframe, maintaining it, and selling the
  computer when they're done.
\end{itemize}

Overall, I think this term is extremely well-defined, considering there is an
ISO specification that is 10 pages long, and is very thorough in defining
everything. It is, after all the international body for defining standards\dots

\subsection{Edge Computing}
According to Cloudflare: 
\begin{displayquote}
  ``
  Edge computing is a networking philosophy focused on bringing computing as close to the source of data as possible in order to reduce latency and bandwidth use. In simpler terms, edge computing means running fewer processes in the cloud and moving those processes to local places, such as on a user’s computer, an IoT device, or an edge server. Bringing computation to the network’s edge minimizes the amount of long-distance communication that has to happen between a client and server.
  ''
\end{displayquote}

Advantages:
\begin{itemize}
  \item Lower latency (better user experience, streaming and gaming applications
    for example. Users would experience higher quality streams and faster
    response times for games)
  \item Less bandwidth and server resource usage, and therefore lower costs
    associated with these resources
\end{itemize}

Overall, it is not very specifically defined. However, most sources seem to
agree that the general idea is moving cloud computing resources geographically
closer to users.

\subsection{Fog Computing}
According to the paper  “Finding your way in the
fog: Towards a comprehensive definition of fog computing” by L. M. Vaquero and
L. Rodero-Merino: 
\begin{displayquote}
  ``Fog computing is a scenario where a huge number of heterogeneous (wireless and sometimes autonomous) ubiquitous
and decentralised devices communicate and potentially cooperate among them and with the network to perform storage and processing tasks without the intervention of thirdparties. These tasks can be for supporting basic network
functions or new services and applications that run in a
sandboxed environment. Users leasing part of their devices
to host these services get incentives for doing so.''
\end{displayquote}


Advantages:
\begin{itemize}
  \item even lower latency and utilizing resources like bandwidth more
    efficiently compared to edge computing. It
    brings the edge closer to the user, kind of like how fog is closer to the
    ground than clouds
  \item allows for data to be sent to a local node for faster proccessing and
    response times. An example application would be sensors in a factory which need to automatically shut down
    equipment if something goes wrong
\end{itemize}

Overall, this term is fairly well-defined. There are multiple papers, IEEE
articles and the like which give in-depth definitions of fog computing which
seem to agree with each other. 

\subsection{Mist Computing}
According to the National Institute of Standards and Technology,
\begin{displayquote}
  ``Mist computing is a lightweight and rudimentary form of fog computing that resides directly within
the network fabric at the edge of the network fabric, bringing the fog computing layer closer to
the smart end-devices. Mist computing uses microcomputers and microcontrollers to feed into fog
computing nodes and potentially onward towards the centralized (cloud) computing
services.''
\end{displayquote}

Advantages:
\begin{itemize}
  \item similar to how fog computing takes edge computing a step further, mist
    computing takes it a step further to reduce latency, and better use
    resources like processing power
  \item no need to connect to the `cloud' to perform decisions, but overall
    telemetry data can still be aggregated by connecting to the broader internet
    Allows for applications like self-driving cars to make decisions locally
    where decisions have to be made quickly. 
\end{itemize}

Overall, this term is not well-defined. Finding a definition was difficult to
begin with. Additionally, the sources that mention mist computing seem to talk
more about fog computing than mist computing. However, generally, they agree on
the local processing aspect of mist computing.
