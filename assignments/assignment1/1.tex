\subsection*{Hard real-time vs. Soft real-time embedded systems}

\subsubsection*{Similarities}
\begin{itemize}
  \item With both, we want to avoid violating real-time constraints
\end{itemize}

\subsubsection*{Differences}
\begin{itemize}
  \item Violations of real-time constraints are undesireable, but tolerated in
    soft real-time systems while in hard real-time systems, violations of
    real-time constraints are unacceptable (a violation for a hard real-time
    system would cause catastrophic failure, or even death.)
\end{itemize}


\subsubsection*{}
When selecting an implementation strategy, it is useful to distinguish between
hard and soft real-time systems because you want to select a strategy that is
appropriate. You want to consider what might happen in the case of failure to 
meet real-time constraints. If a violation of these constraints would cause
catastrophic damage, such as loss of life, you have a hard real-time system,
so your implementation strategy should take that into account. If you have a
soft real time system, and failure to meet the real-time constraints is not
catastrophic, your implementation strategy does not need to avoid
failure to the same extent as it would need to for a hard real-time system.

For example, you probably would have a different implementation strategy for
making a pacemaker versus a vending machine. In one scenario, a person's life is
dependent, while in the other, a person may or may not get a snack.
