\begin{enumerate}
  \item Military applications. Say for example, a missile was being launched,
    and is guided by a computer system. When a missile is launched there were
    (hopefully) multiple humans who made the decision to do so, and they were
    sure of their actions. If the system was able to be overridden by humans,
    that gives the enemy a chance to disarm or take control of the missle, which
    is not ideal. In this case, the missile should lock out human and override
    human control to deliver its payload to the intended recipient.

  \item Antilock braking systems. Humans are not great at pressing on their
    car's brakes just enough so that there is static friction, as opposed to
    kinetic friction. If static friction is maintained, the car can come to a
    stop quicker and also prevents the car from skidding so that the driver can
    steer the car more easily and potentially avoid a collision. In this case,
    the computer forces the brake pedal back up, overriding the human's action
    of pushing the pedal down. This is a feature that has saved many lives.

  \item Safety systems. Imagine a computer at a manufacturing plant which has
    shut down operations due to one of its sensors reporting a value which
    results in an unsafe condition. An uninformed worker, or maybe a
    malpracticing human prioritizing production over safey might try to resume
    operations without fixing the cause of production being shut down. In this
    case, it makes sense for the computer to lock out human control for resuming
    operations until the safety issue is fixed.
\end{enumerate}
